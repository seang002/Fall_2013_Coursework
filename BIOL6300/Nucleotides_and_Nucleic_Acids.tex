\documentclass[a4paper, 10pt]{article}
\usepackage[a4paper,top=1in,bottom=1in,left=1in,right=1in]{geometry}
\usepackage{chemfig}
\usepackage{tabularx}
\usepackage{fancyhdr}
\setlength{\headheight}{1em}
\pagestyle{fancyplain}
\lhead{BIOL6300 - Biochemistry}
\chead{Nucleotides and Nucleic Acids}
\rhead{\today}
\setlength{\parindent}{0cm} % Default is 15pt.
\setlength{\parskip}{1em}

\begin{document}

\textbf{Definitions:}\\
\begin{tabularx}{\linewidth}{lX}
\textbf{RNA} & Ribonucleic Acid\\
\textbf{DNA} & Deoxyribonucleic Acid\\
\textbf{rRNA} & Ribosomal RNA\\
\textbf{tRNA} & Transfer RNA\\
\textbf{mRNA} & Messenger RNA\\
\textbf{gene} & A segment of a DNA molecule that contains a sequence of a functional biological product, whether protein or RNA.\\
\textbf{pyrimidine} & Cytosine, Thymine (DNA) and Uracil (RNA)\\
\textbf{purine} & Adenine and Guanine\\
\end{tabularx}

\definesubmol{ribose}{{^-}O-P(-[2]O\rlap{${}^-$})(=[6]O)(-O-CH_2(-[2,.25,,,draw=none]\scriptstyle\color{red}{5'})-[6]?([7,.7]<(-[2,.4]H)(-[6,.4]OH)(-[1,.25,,,draw=none]\scriptstyle\color{red}3')-[0,.7,,,line width=2pt](-[2,.4]H)(-[6,.4,,,red]\color{red}{OH})>[1,.7](-[6,.4]H-[2,2.5]\framebox{Purine or pyrimidine base})-[:150,.95]O?)-[6]H)}

\definesubmol{deoxyribose}{{^-}O-P(-[2]O\rlap{${}^-$})(=[6]O)(-O-CH_2(-[2,.25,,,draw=none]\scriptstyle\color{red}{5'})-[6]?([7,.7]<(-[2,.4]H)(-[6,.4]OH)(-[1,.25,,,draw=none]\scriptstyle\color{red}3')-[0,.7,,,line width=2pt](-[2,.4]H)(-[6,.4,,,red]\color{red}{H})>[1,.7](-[6,.4]H-[2,2.5]\framebox{Purine or pyrimidine base})-[:150,.95]O?)-[6]H)}

\begin{tabularx}{\linewidth}{cX}
\setcrambond{2pt}{}{}
\chemname{\chemfig{!{ribose}}}{\textbf{Ribose}} &
\setcrambond{2pt}{}{}
\chemname{\chemfig{!{deoxyribose}}}{\textbf{DeoxyRibose}} \\
\end{tabularx}

\definesubmol{pyrimidine}{HC?(-[0,.25,,,draw=none]\color{red}^{2})=[:-30]N(-[2,.25,,,draw=none]\scriptstyle\color{red}{1})-[:30]CH(-[4,.25,,,draw=none]\color{red}^{6})=[:90]CH(-[4,.25,,,draw=none]\color{red}_5)-[:150]\chemabove{C}{H}(-[6,.25,,,draw=none]\scriptstyle\color{red}{4})=[:210]N?(-[0,.25,,,draw=none]\color{red}_3)}

\definesubmol{purine}{HC?[a](-[0,.25,,,draw=none]\color{red}^{2})=[:-30]N(-[2,.25,,,draw=none]\scriptstyle\color{red}{3})-[:30]C(-[4,.25,,,draw=none]\color{red}^{4})(-[:-20]\chembelow{N}{H}(-[2,.25,,,draw=none]\scriptstyle\color{red}{9})-[:60]CH(-[4,.25,,,draw=none]\scriptstyle\color{red}{8})=[:120]N?[b](-[6,.25,,,draw=none]\scriptstyle\color{red}{7}))=[:90]C?[b](-[4,.25,,,draw=none]\color{red}_5)-[:150]\chemabove{C}{H}(-[6,.25,,,draw=none]\scriptstyle\color{red}{6})=[:210]N?[a](-[0,.25,,,draw=none]\color{red}_1)
}

\begin{tabularx}{\linewidth}{XX}
\raisebox{-1.25\height}{\chemname{\chemfig{!{pyrimidine}}}{\textbf{Pyrimidine}}} &
\raisebox{-1.25\height}{\chemname{\chemfig{!{purine}}}{\textbf{Purine}}}\\
\end{tabularx}

\definesubmol{Cytosine}{C?(=[:210]O)(-[0,.25,,,draw=none]\color{red}^{2})-[:-30]\chembelow{N}{H}(-[2,.25,,,draw=none]\scriptstyle\color{red}{1})-[:30]CH(-[4,.25,,,draw=none]\color{red}^{6})=[:90]CH(-[4,.25,,,draw=none]\color{red}_5)-[:150]C(-[2]NH_2)(-[6,.25,,,draw=none]\scriptstyle\color{red}{4})=[:210]N?(-[0,.25,,,draw=none]\color{red}_3)}

\definesubmol{Thymine}{C?(-[0,.25,,,draw=none]\color{red}^{2})(=[:210]O)-[:-30]\chembelow{N}{H}(-[2,.25,,,draw=none]\scriptstyle\color{red}{1})-[:30]CH(-[4,.25,,,draw=none]\color{red}^{6})=[:90]C(-[4,.25,,,draw=none]\color{red}_5)(-[:30]CH_3)-[:150]C(=[2]O)(-[6,.25,,,draw=none]\scriptstyle\color{red}{4})-[:210]HN?(-[0,.25,,,draw=none]\color{red}_3)}

\definesubmol{Uracil}{C?(-[0,.25,,,draw=none]\color{red}^{2})(=[:210]O)-[:-30]\chembelow{N}{H}(-[2,.25,,,draw=none]\scriptstyle\color{red}{1})-[:30]CH(-[4,.25,,,draw=none]\color{red}^{6})=[:90]CH(-[4,.25,,,draw=none]\color{red}_5)-[:150]C(=[2]O)(-[6,.25,,,draw=none]\scriptstyle\color{red}{4})-[:210]HN?(-[0,.25,,,draw=none]\color{red}_3)}


\definesubmol{Adenine}{HC?[a](-[0,.25,,,draw=none]\color{red}^{2})=[:-30]N(-[2,.25,,,draw=none]\scriptstyle\color{red}{3})-[:30]C(-[4,.25,,,draw=none]\color{red}^{4})(-[:-20]\chembelow{N}{H}(-[2,.25,,,draw=none]\scriptstyle\color{red}{9})-[:60]CH(-[4,.25,,,draw=none]\scriptstyle\color{red}{8})=[:120]N?[b](-[6,.25,,,draw=none]\scriptstyle\color{red}{7}))=[:90]C?[b](-[4,.25,,,draw=none]\color{red}_5)-[:150]C(-[6,.25,,,draw=none]\scriptstyle\color{red}{6})(-[2]NH_2)=[:210]N?[a](-[0,.25,,,draw=none]\color{red}_1)
}

\definesubmol{Guanine}{C?[a](-[:210]H_2N)(-[0,.25,,,draw=none]\color{red}^{2})=[:-30]N(-[2,.25,,,draw=none]\scriptstyle\color{red}{3})-[:30]C(-[4,.25,,,draw=none]\color{red}^{4})(-[:-20]\chembelow{N}{H}(-[2,.25,,,draw=none]\scriptstyle\color{red}{9})-[:60]CH(-[4,.25,,,draw=none]\scriptstyle\color{red}{8})=[:120]N?[b](-[6,.25,,,draw=none]\scriptstyle\color{red}{7}))=[:90]C?[b](-[4,.25,,,draw=none]\color{red}_5)-[:150]C(-[6,.25,,,draw=none]\scriptstyle\color{red}{6})(=[2]O)-[:210]HN?[a](-[0,.25,,,draw=none]\color{red}_1)
}

\begin{tabularx}{\linewidth}{XX}
\raisebox{-1.25\height}{\chemname{\chemfig{!{Adenine}}}{\textbf{Adenine}}} &
\raisebox{-1.25\height}{\chemname{\chemfig{!{Guanine}}}{\textbf{Guanine}}} \\
\end{tabularx}
\begin{tabularx}{\linewidth}{XXX}
\chemnameinit{\chemfig{C-[6]H}}
\raisebox{-1.25\height}{\chemname{\chemfig{!{Cytosine}}}{\textbf{Cytosine}}} &
\chemnameinit{\chemfig{C-[6]H}}
\raisebox{-1.25\height}{\chemname{\chemfig{!{Thymine}}}{\textbf{Thymine (DNA)}}} &
\chemnameinit{\chemfig{C-[6]H}}
\raisebox{-1.25\height}{\chemname{\chemfig{!{Uracil}}}{\textbf{Uracil (RNA)}}} \\
\chemnameinit{}
\end{tabularx}
\end{document}