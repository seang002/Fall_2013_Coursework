\documentclass[a4paper, 10pt]{article}
\usepackage[a4paper,top=1in,bottom=1in,left=1in,right=1in]{geometry}
\usepackage{chemfig}
\usepackage{tabularx}
\usepackage{fancyhdr}
\setlength{\headheight}{1em}
\pagestyle{fancyplain}
\lhead{BIOL6300 - Biochemistry}
\chead{Nucleotides and Nucleic Acids}
\rhead{\today}
\setlength{\parindent}{0cm} % Default is 15pt.
\setlength{\parskip}{1em}

\begin{document}

\textbf{Definitions:}\\
\begin{tabularx}{\linewidth}{lX}
\textbf{RNA} & Ribonucleic Acid\\
\textbf{DNA} & Deoxyribonucleic Acid\\
\textbf{rRNA} & Ribosomal RNA\\
\textbf{tRNA} & Transfer RNA\\
\textbf{mRNA} & Messenger RNA\\
\textbf{gene} & A segment of a DNA molecule that contains a sequence of a functional biological product, whether protein or RNA.\\
\textbf{pyrimidine} & Cytosine, Thymine (DNA) and Uracil (RNA)\\
\textbf{purine} & Adenine and Guanine\\
\end{tabularx}

\definesubmol{pyrimidine}{HC?(-[0,.25,,,draw=none]^{2})=[:-30]N(-[2,.25,,,draw=none]\scriptstyle{1})-[:30]CH(-[4,.25,,,draw=none]^{6})=[:90]CH(-[4,.25,,,draw=none]_5)-[:150]\chemabove{C}{H}(-[6,.25,,,draw=none]\scriptstyle{4})=[:210]N?(-[0,.25,,,draw=none]_3)}
\definesubmol{purine}{HC*6(=N-C*5(-\chembelow{N}{H}-CH=N?)=C?-\chemabove{C}{H}=N-)}
\begin{tabularx}{\linewidth}{XX}
\raisebox{-1.5\height}{\chemname{\chemfig{!{pyrimidine}}}{\textbf{Pyrimidine}}} &
\raisebox{-1.5\height}{\chemname{\chemfig{!{purine}}}{\textbf{Purine}}}\\
\end{tabularx}

\raisebox{-1.5\height}{
\setcrambond{2pt}{}{}
\chemname{\chemfig{{^-}O-P(-[2]O\rlap{${}^-$})(=[6]O)(-O-CH_2-[6]?([7,.7]<(-[2,.4]H-[6,.8]OH)-[0,.7,,,line width=2pt](-[2,.4]H-[6,.8]OH)>[1,.7](-[6,.4]H-[2,2.5]\framebox{Purine or pyrimidine base})-[:150,.95]O?)-[6]H)}}{\textbf{Pentose}}}

\end{document}