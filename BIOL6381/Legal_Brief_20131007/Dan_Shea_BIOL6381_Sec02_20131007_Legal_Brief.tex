\documentclass[letterpaper,10pt,twoside]{article}
\usepackage[top=1in,bottom=1in,left=1in,right=1in]{geometry}

%Titlesec lets youformat sections
\usepackage{titlesec}
\titleformat{\section}[block]{\normalfont\bfseries}{\hspace{10mm}\thesection}{0.5em}{}
\titlespacing*{\section}{0pt}{3.5ex plus 1ex minus .2ex}{2.3ex plus.2ex}

%Use URLs
\usepackage{url}
\usepackage{hyperref}
\usepackage[svgnames]{xcolor}
\hypersetup{
  colorlinks,
  urlcolor=Blue}
\urlstyle{same}

% Set color of footnotes to blue
\renewcommand\thefootnote{\textcolor{blue}{\arabic{footnote}}}

% Remove footnote indentation
\usepackage[hang,flushmargin]{footmisc} 

% Add extra blank line between paragraphs, and remove paragraph indentation
%\usepackage[parfill]{parskip}

% Header
\usepackage{fancyhdr}
\pagestyle{fancy}
\lhead{Daniel J. Shea}
\chead{BIOL6381 SEC02 -- Legal Brief}
\rhead{Astrue v. Capato (11-159)}
% Add space between text and footnotes section
\setlength{\skip\footins}{0.25in}

% Creat a new command to handle the title font resizing
\newcommand*{\TitleFont}{%
      \usefont{\encodingdefault}{\rmdefault}{b}{n}%
      \fontsize{12}{1.5em}%
      \selectfont}



\begin{document}
\section*{Case Summary}
Karen and Robert Nicholas Capato were married in Weehawken, New Jersey, on May 15, 1999.  Mr. Capato was soon thereafter diagnosed with esophogeal cancer in August 1999.  In April 2000, Mr. Capato began a program of preserving sperm for in vitro fertilization at the Northwest Center for Infertility and Reproductive Endocrinology in Florida, because the chemotherapy used to treat his cancer could lead to infertility.  On December 12, 2001, Mr. Capato executed his last will and testament under Florida law.  In his will, Mr. Capato provided for his naturally born child with his wife Karen, as well as two children from a previous marriage. The will made no mention of any unborn children or children who may be be conceived through IVF after his death.  On March 23, 2002, Mr. Capato died of metastatic esophageal cancer.  Shortly after her husband's death, in Florida, Karen Capato underwent in vitro fertilization using his frozen sperm.  Mrs. Capato later received further artificial insemination treatments at Reproductive Medicine Associates (``RMA") of New Jersey, starting in November 2002.  Mrs. Capato gave birth to twins on September 23, 2003, eighteen months after the death of her husband.  Dr. Richard Scott of RMA wrote a letter dated December 1, 2003, indicating that the minor children were conceived in January 2003 from the Mr. Capato's sperm.

Shortly after the birth of her twins, Karen Capato applied for Social Security benefits on behalf of her twins under Title II of the Social Security Act, as survivors of a deceased wage earner. The Social Security Administration denied her claim and an administrative law judge confirmed this decision.  The judge ruled that state intestacy\footnote{Intestacy is the condition of the estate of a person who dies owning property greater than the sum of their enforceable debts and funeral expenses without having made a valid will or other binding declaration.} law controls eligibility for survivor benefits for posthumously conceived children under the Social Security Act.  Mrs. Caputo appealed the decision and the district court affirmed the administrative law judge's ruling.  The United States Court of Appeals for the Third Circuit reversed the decision, and ruled that the plain language of the Act entitles the Capato twins, whose parentage is not in dispute, to survivor benefits.  The court of appeals then remanded the case to the district court to determine if the Capato twins were dependents of their father within the meaning of the Act. The Supreme Court granted certiorari\footnote{Certiorari is defined as, ``A writ that the Supreme Court of the United States issues to a lower court to review the lower court's judgement for legal error and review where no appeal is available as a matter of right."} to determine ``whether the biological, posthumous child of married parents who is ineligible to inherit from a deceased parent's estate under state law may nevertheless be entitled to receive survivor benefits under the Act."  The judgement of the Court of Appeals for the Third Circuit finding an entitlement to benefits was reversed, and the case was remanded for further proceedings.

\section*{Opinion}
While the final ruling of the case regarding the Capato twin's eligibility of survivor benefits under the Social Security Act has, as of the time of the authorship of this brief to be completely decided, the case raises several issues surrounding IVF and the posthumous parentage of a child with respect to inheritance law and survivor benefits under the Social Security Act. The administrative law judge used Florida law to define ``child", because \S 416(h)(2)(A)\footnote{See: \url{http://www.ssa.gov/OP_Home/ssact/title02/0216.htm}\\``In determining whether an applicant is the child or parent of a fully or currently insured individual for purposes of this title, the Commissioner of Social Security shall apply such law as would be applied in determining the devolution of intestate personal property by the courts of the State in which such insured individual is domiciled at the time such applicant files application, or, if such insured individual is dead, by the courts of the State in which he was domiciled at the time of his death, or, if such insured individual is or was not so domiciled in any State, by the courts of the District of Columbia. Applicants who according to such law would have the same status relative to taking intestate personal property as a child or parent shall be deemed such."} of the Social Security Act instructs the Commissioner making the determination to use state intestacy laws from ``the State in which [the insured] was domiciled at the time of his death."  However, the Social Security Act's broad definition of what constitutes a ``child", for the purposes of survivor benefits, is sufficiently vague in this regard.  The United States Court of Appeals for the Third Circuit ruled that the broad interpretation of child according to \S 416(e)\footnote{See: \url{http://www.ssa.gov/OP_Home/ssact/title02/0216.htm}, Section (e)}, makes \S 416(h)'s limitations inapplicable because the family status of Mr. Capato's biological offspring was already clear.

It is my belief that the final  decision regarding this case will see further amendments to the Social Security Act with respect to survivor benefits in the case of posthumous IVF parentage.  Much of section \S 416(e) deals with sub-clauses surrounding cases in which a deceased adoptive parent, adopts and supports a child with respect to the decedent's time of death.  There is also a sub-clause surrounding grandparent's in the case that no adoptive or natural parent is living at the time of the grandparent's death.  Following a logical continuation to the line of of reasoning outlined by sub-clauses in \S 416(e), I would expect that a legal framework regarding the time-frame for eligibility of survivor benefits for children born via IVF will be drawn up and added as an amendment to the Social Security Act.

I agree with the initial court's findings that the Capato twins are not deemed eligible under Florida State's intestacy law.  However, I also agree that the Third Circuit was correct in assessing that under the current wording of the Social Security Act, the term ``child" is  sufficiently broad enough to define the Capato twins as children eligible for survivor benefits.  It is my belief that the Third Circuit Court was pressing the issue in furtherance of having the existing Social Security Act amended to clarify such cases, rather than to rely on the court's interpretation of the existing Social Security Act.

It is the author's personal opinion that the IVF procedure should have to be conducted with the knowledge and consent of the decedent prior to his/her death in order for the biological offspring of such a procedure to be deemed eligible for survivor benefits.  Mrs. Capato did not undergo the IVF procedure until after her husband had died.  Therefore, she underwent IVF with the full knowledge that her children did not have Mr. Capato as a wage earner as he was already deceased at the time of the procedure.  Under the wording of the current Social Security Act the Capato twins are eligible for survivor benefits.  I suspect however, that while the Capato twins may, in fact, be found eligible for benefits under the current legal interpretation of the Social Security Act the resulting legal fallout from such a ruling will force legal amendments to restrict future survivor claims in this regard.

\section*{Relevant Case Law}
\begin{enumerate}
\item KAREN K. CAPATO, o/b/o B.N.C., K.N.C., Plaintiff, v. MICHAEL J. ASTRUE, Commissioner of Social Security, Defendant.\footnote{\label{Original Case}KAREN K. CAPATO, o/b/o B.N.C., K.N.C., Plaintiff, v. MICHAEL J. ASTRUE, Commissioner of Social Security, Defendant.\\
Civil Action No.: 08-5405 (DMC)\\
UNITED STATES DISTRICT COURT FOR THE DISTRICT OF NEW JERSEY\\
2010 U.S. Dist. LEXIS 27350}
\item KAREN K. CAPATO, o/b/o B.N.C., K.N.C., Appellant v. COMMISSIONER OF SOCIAL SECURITY\footnote{\label{Appealed}KAREN K. CAPATO, o/b/o B.N.C., K.N.C., Appellant v. COMMISSIONER OF SOCIAL SECURITY\\
No. 10-2027\\
UNITED STATES COURT OF APPEALS FOR THE THIRD CIRCUIT\\
631 F.3d 626; 2011 U.S. App. LEXIS 19; 163 Soc. Sec. Rep. Service 35; Unemployment Ins. Rep. (CCH) P14,886C}
\item MICHAEL J. ASTRUE, COMMISSIONER OF SOCIAL SECURITY, Petitioner v. KAREN K. CAPATO, on behalf of B. N. C., et al.\footnote{\label{US Supreme Court Ruling}MICHAEL J. ASTRUE, COMMISSIONER OF SOCIAL SECURITY, Petitioner v. KAREN K. CAPATO, on behalf of B. N. C., et al.\\
No. 11-159\\
SUPREME COURT OF THE UNITED STATES\\
132 S. Ct. 2021; 182 L. Ed. 2d 887; 2012 U.S. LEXIS 3782; 80 U.S.L.W. 4369; 23 Fla. L. Weekly Fed. S 308; Unemployment Ins. Rep. (CCH) P14,965C; 179 Soc. Sec. Rep. Service 10}
\end{enumerate}
\end{document}