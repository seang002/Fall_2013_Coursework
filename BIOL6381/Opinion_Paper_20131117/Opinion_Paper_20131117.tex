\documentclass[letterpaper,10pt,twoside]{article}
\usepackage[top=1in,bottom=1in,left=1in,right=1in]{geometry}

%Titlesec lets youformat sections
\usepackage{titlesec}
\titleformat{\section}[block]{\normalfont\bfseries}{\hspace{10mm}\thesection}{0.5em}{}
\titlespacing*{\section}{0pt}{3.5ex plus 1ex minus .2ex}{2.3ex plus.2ex}

%Use URLs
\usepackage{url}
\usepackage{hyperref}
\usepackage[svgnames]{xcolor}
\hypersetup{
  colorlinks,
  urlcolor=Blue}
\urlstyle{same}

% Set color of footnotes to blue
\renewcommand\thefootnote{\textcolor{blue}{\arabic{footnote}}}

% Remove footnote indentation
\usepackage[hang,flushmargin]{footmisc} 

% Header
\usepackage{fancyhdr}
\pagestyle{fancy}
\lhead{Dan Shea}
\rhead{BIOL6381 SEC02 -- Opinion Paper}
% Add space between text and footnotes section
\setlength{\skip\footins}{0.25in}

% Create a new command to handle the title font resizing
\newcommand*{\TitleFont}{%
      \usefont{\encodingdefault}{\rmdefault}{b}{n}%
      \fontsize{12}{1.5em}%
      \selectfont}

% Set up double-spacing
\usepackage{setspace}
% \doublespacing
\begin{document}
\begin{center}
\textbf{\large{An argument against ``opt-out" or ``presumed-consent" organ donation laws.}}
\end{center}
\vspace{0.5em}
Currently, most states in the United States have ``opt-in" organ donation laws by which an individual or next of kin must consent to organ donation.  However, prior to the 2006 revision to The Uniform Anatomical Gift Act of 1987\footnote{\url{http://www.uniformlaws.org/shared/docs/anatomical_gift/uaga_final_aug09.pdf}}, states had various differing guidelines regarding the use of cadaver organs for transplant.  There has been some debate concerning once again moving the United States back to an ``opt-out" or ``presumed-consent" based organ donation system to increase participation rates of organ donors.  This false-dichotomy is completely unnecessary.  

Let us examine the main argument in favour of presumed-consent first before examining the alternatives.  Proponents of the system argue that utilizing an ``opt-out" based system dramatically increases organ donor participation rates.  This claim is, in fact, backed by statistical studies of organ donation rates in countries with similar cultures and differing organ donation laws.  Germany and Austria are a prime example of this argument.  In Germany, organ donation consent rate is 12\% the current population, while in Austria, a country with a very similar culture and economic development, but which uses an ``opt-out" system, has a consent rate of 99.98\%.\footnote{ Johnson, Eric J.; and Goldstein, Daniel G. ``Do defaults save lives?" (PDF). Science 302 (5649): 1338–1339. doi:10.1126/science.1091721. PMID 14631022.}  These statistical arguments however only look at organ donor participation rates, and fail to address the moral hazards of instituting such a system.  Furthermore, we shall see that organ donation rates have in fact continued to rise with the current ``opt-in" system that is in place in the majority of states within the United States.  We also propose that this false-dichotomy is improperly presented to the public, as there are other legal frameworks by which one can influence a rise in organ donor participation rates.

Clinical opposition may be raised by looking towards the on-going debate as to what truly constitutes clinical death.  Disagreement within the medical profession regarding at what point a person is no longer considered to be living, muddies the waters regarding determination of irreversible death.  In the United States, the Uniform Determination of Death Act\footnote{\url{http://www.uniformlaws.org/shared/docs/determination\%20of\%20death/udda80.pdf}} defines death as the ``irreversible cessation of the function of either the brain or the heart and lungs."  And within the United States, the 21st century has seen an order-of-magnitude increase of donation following cardiac death.  In 1995, only 1\% of dead donors gave their organs following the declaration of cardiac death.  That figure grew to almost 11\% in 2008.\footnote{\label{When are you dead}John Sanford (Spring 2011). ``When Are You Dead?" Stanford University School of Medicine}  This increase, counters the argument of the need for an ``opt-out" based system, as organ donation rates have increased within the current ``opt-in" system.  It has also raised ethical concerns about the interpretation of ``irreversible" since ``patients may still be alive five or even ten minutes after cardiac arrest because, theoretically, their hearts could be restarted, [and thus are] clearly not dead because their condition was reversible."$^{\ref{When are you dead}}$  

There is also the concern that ``opt-out" systems prey upon the common lack of inaction by the general public to pro-actively deny consent of organ donation.  Requiring individuals to legally assert sovereignty over their body is counter-intuitive to most people's understanding of their inalienable rights.  An ``opt-out" system preys upon this in an unfair manner and disadvantages those without the financial means to legally defend themselves from what is perceived as a violation of [what they rightly believe to be] the fundamental sanctity of their corporeal being, should a conflict over organ donation consent arise.

We must also consider the well documented cases of abuses within the medical profession when examining the pitfall of moving to an ``opt-out" based system.  Japan's organ donation participation rates have been historically much lower than that of other countries.  The lower rates are often attributed to cultural differences.  However, as a permanent legal resident of Japan myself since 2006, my personal experience has been that it is a distrust of the medical profession, not a cultural argument against organ donation.  This distrust stems from some of the earliest cases of organ transplantation in Japan.  The first heart transplant in Japan was conducted at Sapporo Medical University in 1968 by Dr. Wada.  This operation attracted serious concerns surrounding Dr. Wada's evaluation of brain death.  Many felt at the time that this evaluation was inappropriate. While an investigation of possible criminal liability was dismissed, a societal distrust of the medical community surrounding organ transplantation developed as a result, particularly of transplants from brain dead donors. This public distrust brought about a marked decline in organ donor participation rates and a halting of subsequent transplantations using clinically brain dead donors.\footnote{``Organ Transplantation and Brain-Death in Japan." \url{http://www.bioethics.jp/licht_transplant98.html}}

In 2008, California transplant surgeon Hootan Roozrokh was charged with dependent adult abuse for prescribing what prosecutors alleged were excessive doses of morphine and sedatives to hasten the death of a man with adrenal leukodystrophy and irreversible brain damage, in order to procure his organs for transplant. The case brought against Roozrokh was the first criminal case against a transplant surgeon in the US, and resulted in his acquittal.\footnote{Jesse McKinley (2008-02-27). ``Surgeon Accused of Speeding a Death to Get Organs" The New York Times (The New York Times Company)}

At California's Emanuel Medical Center, neurologist Narges Pazouki, MD, said an organ-procurement organization representative pressed her to declare a patient brain-dead before the appropriate tests had been done.  She subsequently refused.  Of particular note is the following statement from medical doctor Michael A. DeVita regarding undue pressures such organ-procurement organizations may resort to, in their efforts to increase access to viable human organs. ```This is what we've been worrying about,' said Michael A. DeVita, a University of Pittsburgh critical care specialist. `If you promote organ donation too much, people lose sight that it's a dying patient there. It's not just a source of organs. It's a person.'"\footnote{Rob Stein (2007-09-13). ``New Zeal in Organ Procurement Raises Fears" The Washington Post (The Washington Post Company)}

While such cases are thankfully rare, they raise the level of distrust the public places upon the medical profession to merely declaring them clinically dead in order to harvest their organs for transplantation  in lieu of providing them with critically needed care.  We cannot merely ignore the public's perception of the medical community and documented cases of abuses by those involved in organ donations when considering a move to ``opt-out" based organ donor laws.

If we truly wish to further improve organ donation rates amongst the population, we do not need to rely on ``opt-out" based laws.  We merely need look towards other countries and the state of Illinois' approaches to the issue of organ donation to see what successes they have had in increasing organ donor participation rates.

Israel takes a unique approach regarding organ donation, by instilling additional medical benefit upon those who choose to ``opt-in."  Israel, realizing that classifying oneself as an organ donor presents an undisputed benefit to society as a whole, has determined that in the event of two individuals, both in equal medical need of an organ transplant, where one is a registered organ donor an the other is not, preference for organ transplantation will be given to the registered organ donor.  The resulting increase in organ donation participation rates was evidenced upon passing this preferential treatment into law in 2008.  This idea of mutually beneficial cooperation by the individual and the medical community furthers trust in the system and provides clearly drawn upon benefits to those who ``opt-in" to the organ donation system.

There is also another possibility referred to as ``mandated-choice."  Under this system, people must indicate their preference.  In Illinois, this system has been in use since 2006 and has not raised any strenuous moral objections.  When an individual renews their driver's license, they are required to answer the following question: ``Do you wish to be an organ donor?"  Illinois now has a 60 percent donor sign-up rate, according to Donate Life Illinois.  Illinois now has a registered organ donor rate much higher than the national rate of 38 percent, as reported by Donate Life America.\footnote{Richard H. Thaler (September 26, 2009). ``Opting in vs. Opting Out" The New York Times (The New York Times Company)}

The Illinois system of mandated-choice has another additional advantage.  There can be legal conflicts over whether registering intent is enough to qualify an individual as an organ donor or whether a doctor must still ask for the family's consent. For example, in France, although there is an ``opt-out" law based on presumed-consent, doctors still seek relatives' approval prior to organ donation.  Illinois' First-Person Consent Law however, makes the individual's wishes to be a donor legally binding.  A mandated-choice based  law may achieve a higher rate of donations than a presumed-consent, ``opt-out" based law.  It also avoids entirely the debate surrounding an ``opt-out" based system.

Ultimately, the trust of the people in its institutions providing medical care, is the ultimate factor in increasing organ donor participation rates.  If the medical profession conducts itself in a manner that inspires the trust of the people, there is less resistance to organ donation participation.  Brazil is a recent example of a nation that failed to inspire such confidence, and as a result had to reverse a switch to an ``opt-out" system because it further alienated patients who already distrust the country's medical system.\footnote{``Time to move to presumed consent for organ donation" Press Release. BMJ Publishing Group.}  If the medical profession within the United States truly wishes to further alienate themselves from the general public, then it would appear that moving to a formal ``opt-out" based system of organ donation would certainly aid them in that regard.  However, we can forgo the entire debate by moving towards a ``mandated-choice" based approach that everyone should be able to agree upon.
\end{document}