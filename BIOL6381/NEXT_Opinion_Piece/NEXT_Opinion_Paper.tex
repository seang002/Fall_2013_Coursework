\documentclass[letterpaper,10pt,twoside]{article}
\usepackage[top=1in,bottom=1in,left=1in,right=1in]{geometry}

%Titlesec lets youformat sections
\usepackage{titlesec}
\titleformat{\section}[block]{\normalfont\bfseries}{\hspace{10mm}\thesection}{0.5em}{}
\titlespacing*{\section}{0pt}{3.5ex plus 1ex minus .2ex}{2.3ex plus.2ex}

%Use URLs
\usepackage{url}
\usepackage{hyperref}
\usepackage[svgnames]{xcolor}
\hypersetup{
  colorlinks,
  urlcolor=Blue}
\urlstyle{same}

% Set color of footnotes to blue
\renewcommand\thefootnote{\textcolor{blue}{\arabic{footnote}}}

% Remove footnote indentation
\usepackage[hang,flushmargin]{footmisc} 

% Header
\usepackage{fancyhdr}
\pagestyle{fancy}
\lhead{Dan Shea}
\rhead{BIOL6381 SEC02 -- NEXT Opinion Paper}
% Add space between text and footnotes section
\setlength{\skip\footins}{0.25in}

% Create a new command to handle the title font resizing
\newcommand*{\TitleFont}{%
      \usefont{\encodingdefault}{\rmdefault}{b}{n}%
      \fontsize{12}{1.5em}%
      \selectfont}

% Set up double-spacing
\usepackage{setspace}
% \doublespacing
\begin{document}
\begin{center}
\textbf{\large{The ethical and legal consequences of the creation of a human-chimpanzee hybrid like ``Dave"--A character from Michael Crichton's novel Next.}}
\end{center}
\vspace{0.5em}

One of the elements of Michael Crichton's book ``Next" that I found to be of particular ethical significance was Henry Kendall's work at the NIH that results in the creation of a transgenic chimpanzee named Dave.  Dave, a transgenic human-chimpanzee hybrid, was the result of an unauthorized experiment that Henry carried out while he was a researcher at the NIH.  Henry utilized recombinant DNA techniques, using his own genetic materials in attempts to create a transgenic foetus in furtherance of his own research.  Expecting the mother of the foetus to spontaneously abort the offspring, Henry finds out later that the mother carried Dave to term and gave birth to him while in quarantine (due to a meningitis outbreak at the NIH primate research facility.)  There are several ethical and legal issues that this scenario presents.  I would like to set aside the ethical violations that Henry's character commits in order to create the transgenic chimpanzee Dave, in order to address the central theme of the ethical and legal issues surrounding the ability of science being able to create transgenic animals and the resulting consequences that will no doubt eventually arise when such an animal is created in reality.

It is my belief that it is only a matter of time before the science behind creating a creature such as Dave is a reality.  Scientists have experimented with the idea of creating a hybridized ``humanzee" as early as 1910 with the Ivanov experiments.  Dr. Ilya Ivanovich Ivanov gave a presentation to the World Congress of Zoologists in Graz, Austria.  He presented ideas on artificial insemination techniques to attempt the creation of a human-ape hybrid.  In the 1920's he conducted research into artificial insemination of chimpanzees through the use of his own sperm and via his son Illya's sperm.\footnote{Rossiianov, Kirill (2002). "Beyond species: Il'ya Ivanov and his experiments on cross-breeding humans with anthropoid apes". Science in Context 15 (2): 277–316.}  All three attempts at artificial insemination of female chimpanzees failed to produce a pregnancy.  His later efforts surrounded the use of male chimpanzee sperm to inseminate a human female volunteer and were planned in the Soviet Union in 1927.  Ivanov was attempting to procure sexually mature male chimpanzees, but was unable to conduct these experiments before his falling out with Soviet political leaders. This lead to his exile to Kazakh, where he died 2 years later.  Interestingly enough, he was however able to find a female volunteer for such an experiment, who is only recorded as ``G" in documentation regarding the planned experiments.  J. Michael Bedford later showed that it was possible for human sperm to penetrate the protective outer membranes of a gibbon egg.  His work, conducted in 1977 showed that the specificity of human spermatozoa was not restricted to humans alone, and that in theory, hybridization may be possible with primates of the Hominoidea Superfamily.    Further research into the idea of creating a human hybrid was carried out in 2006.  It has been shown that the last common ancestor of humans and chimpanzees diverged  into two distinct lineages but that inter-lineage sex was still prevalent enough to produce fertile hybrids for roughly 1.2 million years after the initial split.\footnote{Wade, Nicholas. "Two Splits Between Human and Chimp Lines Suggested", The New York Times, 18 May 2006.}

While there has never been a confirmed human-chimpanzee hybrid in the current body of scientific knowledge, there is sufficient interest in the idea even within current scientific community that research is still being conducted.  The state of the art, regarding modern recombinant DNA techniques, far outstrips the fertilization techniques applied by Ivanov in the 1920's and while the feasibility of such a hybridization is still being debated, researchers are moving forward with investigation into the topic.  This raises some fairly significant ethical and moral questions (along with a myriad of legal issues surrounding what rights a hybrid would be afforded under the law) that we should attempt to address before the creation of the first hybrid.  There is no doubt that religious and moral objections will be raised surrounding the creation of such a transgenic animal.  However, assuming a hybrid is created, what rights and what responsibilities do we have regarding being both the father and the mother of such an offspring?  I would argue that regardless of moral objections some may raise, it is almost certainly an inevitability that the world will one day see such hybridized animals if they are scientifically feasible, therefore it is our duty as the creators of such an organism to afford it rights and protections under the law.

The issue of slavery must be addressed in order to prevent an underclass of transgenic animals being created solely for the purpose of manual labour or for use as military conscripts.  Crichton alludes to this when he references a farmer, who upon seeing Dave, exclaims he would ``love to have one of them."  If we were to create transgenic animals capable of higher cognitive function we have a moral and ethical obligation to treat them with compassion and afford them inalienable rights and protection under the law from abuse or exploitation by those that would have no moral or ethical objections to using them for their own means and ends.  The creation of Dave in the novel shows that in spite of Dr. Kendall's ethical failings that lead to the creation of Dave, he rightly sees him as his own son and affords him the love and caring that a father gives to their own son.  This parental instinct shown by Henry and his wife when confronted with the existence of Dave struck a chord with me as I read the book.  It also gave me pause for consideration as to the tremendous responsibility we have to any transgenic animals we may create in the future.  As a parent of two boys myself, I can relate to the enormous feelings of responsibility that one has for their own offspring, and as such, I feel that if we are to embark upon the road of scientific endeavour that leads us to the creation of animals capable of human-like higher order cognitive function, it is not enough that we merely afford them protection from cruelty as we do with other animals.  We must ensure that they are not exploited as ``beasts of burden."  Just as we do not legally allow the enslavement of other human beings in our society.

There is some room for debate as to what rights they can be afforded under the law.  For example, should Dave be allowed to vote?  Is he held to the same legal standards as any other child his age when he reacts violently?  I suspect until we actually determine what the functional cognition of such a hybrid entails, we can only speculate as to how far we will have to go when it comes to equal rights under the law.  Therefore, I would propose that any transgenic animal created to show cognitive function comparable to a human being, as determined by application of intelligence testing outlined by Cattell-Horn-Carroll\footnote{Carroll, J.B. (1993). Human cognitive abilities: A survey of factor-analytic studies. Cambridge, England: Cambridge University Press.}\footnote{Cattell, R. B. (1941). Some theoretical issues in adult intelligence testing. Psychological Bulletin, 38, 592.}\footnote{Horn, J. L. (1965). Fluid and crystallized intelligence: A factor analytic and developmental study of the structure among primary mental abilities. Unpublished doctoral dissertation, University of Illinois, Champaign.\\See: \url{http://www.iapsych.com/articles/horn1965.pdf}} should have the same rights and privileges in our society as afforded human beings.

Cattell-Horn-Carroll's cognitive model covers a broad range of factors, best suited to determining the various intelligences of an individual.  Currently, there are 9 broad stratum abilities and over 70 narrow abilities below these, as defined by current tests based on this theory.  The broad abilities are defined below.
\begin{itemize}
\item Crystallized Intelligence (Gc): includes the breadth and depth of a person's acquired knowledge, the ability to communicate one's knowledge, and the ability to reason using previously learned experiences or procedures.
\item Fluid Intelligence (Gf): includes the broad ability to reason, form concepts, and solve problems using unfamiliar information or novel procedures.
\item Quantitative Reasoning (Gq): is the ability to comprehend quantitative concepts and relationships and to manipulate numerical symbols.
\item Reading and Writing Ability (Grw): includes basic reading and writing skills.
\item Short-Term Memory (Gsm): is the ability to apprehend and hold information in immediate awareness and then use it within a few seconds.
\item Long-Term Storage and Retrieval (Glr): is the ability to store information and fluently retrieve it later in the process of thinking.
\item Visual Processing (Gv): is the ability to perceive, analyze, synthesize, and think with visual patterns, including the ability to store and recall visual representations.
\item Auditory Processing (Ga): is the ability to analyze, synthesize, and discriminate auditory stimuli, including the ability to process and discriminate speech sounds that may be presented under distorted conditions.
\item Processing Speed (Gs): is the ability to perform automatic cognitive tasks, particularly when measured under pressure to maintain focused attention.
\end{itemize}
A tenth ability, Gt, is also considered part of the theory.  However, it does not appear in cross-battery reference materials, and is not currently assessed by any major intellectual ability test.  
\begin{itemize}
\item Decision/Reaction Time/Speed (Gt): reflects the immediacy with which an individual can react to stimuli or a task (typically measured in seconds or fractions of seconds; not to be confused with Gs, which typically is measured in intervals of 2–3 minutes.)
\end{itemize}

According to current cognitive theory, it may be argued that any transgenic animal that scores within the acceptable human ranges of the testing criteria listed above has a comparable cognitive ability to that of a human.  We can infer that such an animal would be capable of functioning within our society and be able to contribute as a functioning member within human society.  Obviously testing each and every animal on a case by case basis will not be possible.  Testing would need to be conducted on a statistically significant population of such transgenic individuals and the resulting scores applied much in the same way we currently draw distinctions within our own species for the purposes of drafting legal exemptions to those considered to be mentally incapable of understanding the consequences of their own actions.

Given the broad reaching scope such a change would have upon society and the implications this would potentially have to humanity as a whole, we should proceed cautiously when examining this theory.  It is however, our ethical obligation to consider such a theory.  The ramifications of creating an intelligent transgenic animal affect our very ideas of humanity, society and the very foundations of our legal system.  Some would go so far as to argue that, for this very reason, the creation of higher cognitive transgenic animals should be outlawed, lest in our attempts to play God we fly too close to the sun and, like Icarus of ancient Greek mythology, find our wax wings have melted as we plunge back to earth.

I would however, like to counter that argument.  We have a legal framework capable of dealing with new ideas, new inventions and even new forms of life.  To retard the progress of scientific endeavour out of fear of the unknown is not a path upon which we should embark.  It is the very antithesis of the scientific ethos.  Much like the book's character Dr. Bellarmino, I believe that a divine creator that has given us the capabilities to shape life itself would not be opposed to us utilizing these abilities.  It our responsibility to ensure that we behave with prudence, morality and an ethical responsibility in our actions.  Appeals to religious prohibitions historically never have been successful at curbing mankind's desire and ability to delve into the unknown and seek out knowledge.  Our own religious texts reference this in the parable of the Garden of Eden.  Adam and Eve partake of the forbidden fruit in spite of being told by God himself not to do so.  This is the consequence, and responsibility, of free will.

Therefore, we should show the foresight to begin having this conversation within our scientific and legal communities now. When the first human-hybridized or human-transgenic animals with higher cognitive function are living among us, we should know how to appropriately deal with caring for them and affording them protection from those in our society who would no doubt be of a less moral character with respect to these individuals.  There is no moral ambiguity that as a creation of mankind, it is our responsibility to our ``children" to ensure they are not abused or exploited for financial gain.  Most importantly, we must start to give serious consideration now as to what course of action we should take with respect to the rights and protections afforded to human-hybrid transgenic animals.
\paragraph{References}
\begin{itemize}
\item Elisabeth H. Ormandy, Julie Dale, and Gilly Griffin (2011). Genetic engineering of animals: Ethical issues, including welfare concerns. Can Vet J. 2011 May; 52(5): 544–550. PMCID: PMC3078015 \url{http://www.ncbi.nlm.nih.gov/pmc/articles/PMC3078015/}
\item Stephanie Pain (2008).  ``Blasts from the past: The Soviet ape-man scandal." New Scientist, Magazine issue 2670\\ \url{http://www.newscientist.com/article/mg19926701.000-the-forgotten-scandal-of-the-soviet-apeman.html}
\item Jeff Hecht (2003).``Chimps are human, gene study implies." New Scientist, website\\ \url{http://www.newscientist.com/article.ns?id=dn3744}
\end{itemize}
\end{document}