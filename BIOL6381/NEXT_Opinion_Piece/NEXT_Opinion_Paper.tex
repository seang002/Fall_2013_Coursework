\documentclass[letterpaper,10pt,twoside]{article}
\usepackage[top=1in,bottom=1in,left=1in,right=1in]{geometry}

%Titlesec lets youformat sections
\usepackage{titlesec}
\titleformat{\section}[block]{\normalfont\bfseries}{\hspace{10mm}\thesection}{0.5em}{}
\titlespacing*{\section}{0pt}{3.5ex plus 1ex minus .2ex}{2.3ex plus.2ex}

%Use URLs
\usepackage{url}
\usepackage{hyperref}
\usepackage[svgnames]{xcolor}
\hypersetup{
  colorlinks,
  urlcolor=Blue}
\urlstyle{same}

% Set color of footnotes to blue
\renewcommand\thefootnote{\textcolor{blue}{\arabic{footnote}}}

% Remove footnote indentation
\usepackage[hang,flushmargin]{footmisc} 

% Header
\usepackage{fancyhdr}
\pagestyle{fancy}
\lhead{Dan Shea, Max Hume, Raga Vadhi}
\rhead{BIOL6381 SEC02 -- NEXT Opinion Paper}
% Add space between text and footnotes section
\setlength{\skip\footins}{0.25in}

% Create a new command to handle the title font resizing
\newcommand*{\TitleFont}{%
      \usefont{\encodingdefault}{\rmdefault}{b}{n}%
      \fontsize{12}{1.5em}%
      \selectfont}

% Set up double-spacing
\usepackage{setspace}
\doublespacing
\begin{document}
One of the elements of Michael Crichton's book ``Next" that I found to be of particular ethical significance was Henry Kendall's work at the NIH that results in the creation of a transgenic chimpanzee named Dave.  Dave was the result of an unauthorized experiment that Henry carried out while he was a researcher at the NIH.  Henry utilized recombinant DNA techniques, using his own genetic materials in attempts to create a transgenic foetus in furtherance of his own research.  Expecting the mother of the foetus to spontaneously abort the offspring, Henry finds out later that the mother carried Dave to term and gave birth to him while in quarantine (due to a meningitis outbreak at the NIH primate research facility.)  There are several ethical issues that this scenario presents, however I would like to set aside the ethical violations that Henry's character commits in order to create the transgenic chimpanzee Dave, in order to address the central theme of the ethical issues surrounding the ability of science being able to create transgenic animals and the resulting consequences that will no doubt eventually arise when such an animal is created in reality.

It is my belief that it is only a matter of time before the science behind creating a creature such as Dave is a reality.  Scientists have experimented with the idea of creating a hybridized ``humanzee" as early as 1910 with the Ivanov experiments.  Dr. Ilya Ivanovich Ivanov gave a presentation to the World Congress of Zoologists in Graz, Austria.  He presented ideas on artificial insemination techniques to attempt the creation of a human-ape hybrid.  In the 1920's he conducted research into artificial insemination of chimpanzees through the use of his own sperm and via his son Illya's sperm.  All three attempts at artificial insemination of female chimpanzees failed to produce a pregnancy.  His later efforts surrounded the use of male chimpanzee sperm to inseminate a human female volunteer and were planned in the Soviet Union in 1927.  Ivanov was attempting to procure sexually mature male chimpanzees, but was unable to conduct these experiments before his falling out with Soviet political leaders. This lead to his exile to Kazakh, where he died 2 years later.  Interestingly enough, he was however able to find a female volunteer for such an experiment, who is only recorded as ``G" in documentation regarding the planned experiments.  J. Michael Bedford later showed that it was possible for human sperm to penetrate the protective outer membranes of a gibbon egg.  His work, conducted in 1977 showed that the specificity of human spermatozoa was not restricted to humans alone, and that in theory, hybridization may be possible with primates of the Hominoidea Superfamily.    Further research into the idea of creating a human hybrid was carried out in 2006.  It has been shown that the last common ancestor of humans and chimpanzees diverged  into two distinct lineages but that inter-lineage sex was still prevalent enough to produce fertile hybrids for roughly 1.2 million years after the initial split.

While there has never been a confirmed human-chimpanzee hybrid in the current body of scientific knowledge, there is sufficient interest in the idea even within current scientific community that research is still being conducted.  The state of the art, regarding modern recombinant DNA techniques, far outstrips the fertilization techniques applied by Ivanov in the 1920's and while the feasibility of such a hybridization is still being debated, researchers are moving forward with investigation into the topic.  This raises some fairly significant ethical and moral questions (along with a myriad of legal issues surrounding what rights a hybrid would be afforded under the law) that we should attempt to address before the creation of the first hybrid.  There is no doubt that religious and moral objections will be raised surrounding the creation of such a transgenic animal.  However, assuming a hybrid is created, what rights and what responsibilities do we have regarding being both the father and the mother of such an offspring?  I would argue that regardless of moral objections some may raise, it is almost certainly an inevitability that the world will one day see such hybridized animals if they are scientifically feasible and therefore it is our duty as the creators of such an organism to afford it rights and protections under the law.

The issue of slavery must be addressed in order to prevent an underclass of transgenic animals being created solely for the purpose of manual labour or for use as military conscripts.  Crichton alludes to this when he references a farmer, who upon seeing Dave, exclaims he would ``love to have one of them."  If we were to create transgenic animals capable of higher cognitive function we have a moral and ethical obligation to treat them with compassion and afford them inalienable rights and protection under the law from abuse or exploitation by those that would have no moral or ethical objections to using them for their own means and ends.  The creation of Dave in the novel shows that in spite of Dr. Kendall's ethical failings that lead to the creation of Dave, he rightly sees him as his own son and affords him the love and caring that a father gives to their own son.  This parental instinct shown by Henry and his wife when confronted with the existence of Dave struck a chord with me as I read the book.  It also gave me pause for consideration as to the tremendous responsibility we have to any transgenic animals we may create in the future.  As a parent of two boys myself, I can relate to the enormous feelings of responsibility that one has for their own offspring, and as such, I feel that if we are to embark upon the road of scientific endeavour that leads us to the creation of animals capable of human-like higher order cognitive function, it is not enough that we merely afford them protection from cruelty as we do with other animals.  We must ensure that they are not exploited as ``beasts of burden."  Just as we do not legally allow the enslavement of other human beings in our society.

There is some room for debate as to what rights they can be afforded under the law.  For example, should Dave be allowed to vote?  Is he held to the same legal standards as any other child his age when he reacts violently?  I suspect until we actually determine what the functional cognition of such a hybrid entails, we can only speculate as to how far we will have to go when it comes to equal rights under the law.  However, there is no moral ambiguity that as a creation of mankind, it is our responsibility to our ``children" to ensure they are not abused or exploited for financial gain.  Most importantly, we must start to give serious consideration as to what course of action we should take with respect to the rights and protections afforded to human-hybrid transgenic animals.  When the first human-hybridized or human-transgenic animals with higher cognitive function are living among us, we should know how to appropriately deal with caring for them and affording them protection from those in our society who would no doubt be of a less moral character with respect to these individuals.
\end{document}