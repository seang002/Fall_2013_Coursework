\documentclass[letterpaper,10pt,twoside]{article}
\usepackage[top=1in,bottom=1in,left=1in,right=1in]{geometry}

%Titlesec lets youformat sections
\usepackage{titlesec}
\titleformat{\section}[block]{\normalfont\bfseries}{\hspace{10mm}\thesection}{0.5em}{}
\titlespacing*{\section}{0pt}{3.5ex plus 1ex minus .2ex}{2.3ex plus.2ex}

%Use URLs
\usepackage{url}
\usepackage{hyperref}
\usepackage[svgnames]{xcolor}
\hypersetup{
  colorlinks,
  urlcolor=Blue}
\urlstyle{same}

% Set color of footnotes to blue
\renewcommand\thefootnote{\textcolor{blue}{\arabic{footnote}}}

% Remove footnote indentation
\usepackage[hang,flushmargin]{footmisc} 

% Header
\usepackage{fancyhdr}
\pagestyle{fancy}
\lhead{Dan Shea}
\rhead{BIOL6381 SEC02 -- Final Project}
% Add space between text and footnotes section
\setlength{\skip\footins}{0.25in}

% Create a new command to handle the title font resizing
\newcommand*{\TitleFont}{%
      \usefont{\encodingdefault}{\rmdefault}{b}{n}%
      \fontsize{12}{1.5em}%
      \selectfont}

% Set up double-spacing
\usepackage{setspace}
\doublespacing
\begin{document}
\begin{center}
\textbf{\large{The role of the scientific community to the public regarding genetically modified organisms in agriculture.}}
\end{center}
%\vspace{0.25em}
I have chosen to pursue a Master of Science degree in Bioinformatics, culminating with a PhD in Genetics because of my interest in alleviating human suffering and promoting advances in science through the study and application of Genomics to improve crop yields of the rice strain Oryza Sativa Japonica.  The central theme of my research interest is in the cultivation of strains of rice that will yield agronomically viable crop yields when placed under stress from nutrient depleted soils.  Rice is a cereal crop that currently provides a main source of nutrition for roughly half of the world's current population.  The dependence on petroleum-based fertilizer inputs in agriculture have caused food prices to rise as the world's petroleum production becomes unable to keep pace with demand for petroleum products.  

Predicated upon data arising from resource depletion modelling, and coupled with increasing resolution in that data regarding world petroleum reserves and other natural resource reserves, starting in 2004, I decided to explore various aspects of modern society that would be affected in this era of declining petroleum and energy resources.  One very important aspect of declining petroleum resources is the role that these resources play in modern agriculture.  Indirect consumption of mainly oil and natural gas used to manufacture fertilizers and pesticides accounted for 0.6 quadrillion BTU in 2002.\footnote{\label{Schnepf}Schnepf, Randy (19 November 2004). \href{http://www.nationalaglawcenter.org/wp-content/uploads/assets/crs/RL32677.pdf}{``Energy use in Agriculture: Background and Issues"} CRS Report for Congress. Congressional Research Service.}  In a report published by The Royal Society in 2010, agriculture is becoming increasingly dependent on the direct and indirect input of fossil fuels.\footnote{Jeremy Woods, Adrian Williams, John K. Hughes, Mairi Black and Richard Murphy (August 2010) \href{http://rstb.royalsocietypublishing.org/content/365/1554/2991.full}{``Energy and the food system"} Philosophical Transactions of the Royal Society 365 (1554): 2991–3006.}  It is my goal to improve the economic situation of farmers faced with rising fertilizer prices and consumers faced with rising food prices by applying modern scientific techniques into the creation of food crops that will reduce the need for petroleum inputs into the agricultural industry and enable us to improve access to nutritional resources worldwide.

This leads us to the on-going concerns many in the public have with GM foods.  Currently, in the United States, there is no labeling requirement for GM food products.  However, a 2013 poll by the New York Times showed that 93\% of Americans want GMO labeling.\footnote{Allison Kopicki for the New York Times. July 27, 2013 \href{http://www.nytimes.com/2013/07/28/science/strong-support-for-labeling-modified-foods.html?_r=0}{Strong Support for Labeling Modified Foods}}  It has been a mistake on the part of the FDA and the agricultural industry to not support and require the labeling of GM food products.  There is still an on-going debate within the scientific community as to the safety of such foods.  The European Scientists for Social and Environmental Responsibilities (ENSSER) posted a statement in October of 2013, claiming that there is no scientific consensus on the safety of GM foods.\footnote{\href{http://www.ensser.org/increasing-public-information/no-scientific-consensus-on-gmo-safety/}{Statement: No scientific consensus on GMO safety}, ENSSER, 10/21/2013}  It was signed by about 200 scientists in various fields within its first week.  While critics of ENSSER refer to the organization as an ``anti-GMO activist group", we must continue to have an open and honest debate regarding the safety of such products.

Unfortunately, it appears that the scientific community has not been up to the task of public relations when it comes to explaining the science behind GM foods to non-scientists.  This leads to the public not trusting the scientific community when new discoveries are made that can better society.  One only needs to look at declining child vaccination rates\footnote{ David Ropeik (01 August 2011) \href{http://contemporarypediatrics.modernmedicine.com/contemporary-pediatrics/news/modernmedicine/modern-medicine-now/declining-vaccination-rates}{``Declining vaccination rates"}Contemporary Pediatrics} to see another area where we as scientists have failed to be open and up-front in disclosure surrounding scientific advancements.  While we already have in place a regulatory and legal framework surrounding the application of genetically modified organisms for use in food products, it is incumbent upon the scientific community to ensure the safety of such products and to put forward the case of why these crops are safe for human consumption.  We must also educate the public regarding the creation and use of such products.  The legal consequences of releasing a potentially harmful product to an unsuspecting public would result in complete loss of trust in the scientific community by the public and possibly civil and/or criminal litigation of those involved.  One could argue that GM food researchers have the same legal responsibilities and liabilities when it comes to GM food products as those in the pharmaceutical industry have with respect to new medicines and vaccines when it comes to safety and potential legal liability.  This line of reasoning is applicable to the argument for the labeling of GM food products.  The consumer must be allowed to make an informed decision and assess what risks, if any, they feel comfortable with when it comes to making use of GMO food products in their diets.  While labeling such GM foods does not absolve scientists of responsibility regarding ensuring that products are safe, it adds an element of informed consent.  It reduces the ability of potential litigation based on the claim that individuals unknowingly ingested GM foods and were not made aware of any potential risks such products may or may not have with respect to their use in an individual's diet.

It is my hope and my intention, that through my research and my actions as a member of the scientific community to improve the level of trust in our relationship with the public.  To assist in the education of the public surrounding GM food crops, and to alleviate fears and concerns around such crops in cases where such concern is not warranted.  As members of the scientific community, we must hold ourselves up to a higher standard.  It is our duty to ensure that such GM food products are, in fact, safe for public consumption before releasing such products into the biosphere and the food markets of the world.  Most importantly, we must remember that in serving the public good, we must foster an open relationship with the public and adequately communicate our discoveries to the public at large.  It is through our deeds and words that we will encourage adoption of new technologies by presenting a clear and compelling case for the application of such discoveries for the greater benefit of humanity.
\end{document}